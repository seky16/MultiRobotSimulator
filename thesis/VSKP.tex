%%%
%%%
%%% Vytvoří náležitosti dle směrnice rektora
%%%    Soubor VSKP.tex
%%%
%% Do preambule hlavního souboru vložte následující příkazy:
%%    \usepackage{VSKP}  % Načte styl šablony dle směrnice rektora
%%    %%%
%%%
%%% Vytvoří náležitosti dle směrnice rektora
%%%    Soubor VSKP.tex
%%%
%% Do preambule hlavního souboru vložte následující příkazy:
%%    \usepackage{VSKP}  % Načte styl šablony dle směrnice rektora
%%    %%%
%%%
%%% Vytvoří náležitosti dle směrnice rektora
%%%    Soubor VSKP.tex
%%%
%% Do preambule hlavního souboru vložte následující příkazy:
%%    \usepackage{VSKP}  % Načte styl šablony dle směrnice rektora
%%    %%%
%%%
%%% Vytvoří náležitosti dle směrnice rektora
%%%    Soubor VSKP.tex
%%%
%% Do preambule hlavního souboru vložte následující příkazy:
%%    \usepackage{VSKP}  % Načte styl šablony dle směrnice rektora
%%    \input{VSKP}       % Načte data pro vyplnění šablony
%%    \usepackage{fontspec}  % Pro vkládání OTF fontů (vyžaduje titulní list) - nefunguje v pdfLaTeXu
%%
%% Na začátek hlavního souboru (za \begin{document} ) vložte příkazy pro vysázení desek
%%   \titul% vytiskne titul práce
%%   \abstrakty% vytiskne stránku s abstrakty
%%
%% Použité kódování UTF-8
%%
%%% Generování údajů

\fakulta{Fakulta strojního inženýrství}
\enfakulta{Faculty of Mechanical Engineering}
\adresafakulta{Technická 2896/2, 61669 Brno}

\ustav{Ústav automatizace a informatiky}
\enustav{Institute of Automation and Computer Science}

% udaje o autorovi

\autor{Bc.}{Ondřej Sekáč}{}  % Jméno autora, 
    % Tituly vložte samostatně, např. \autor{Ing.}{Petra Smékalová}{}
\autorzkr{Sekáč, O. }
  % bibliografické jméno

\typstudia{N}
  % M, N, B, D
  % M - Magisterské, N - Navazující magisterské, B - Bakalářské, D-Doktorské
  % U typu studia M a N se liší anglický název

\nazev{Plánování cesty pro více robotů} 
  % Ručně můžete dlouhý text zalomit pomocí " \break "
\ennazev{Path planning for multiple robots} 
  % Ručně můžete dlouhý text zalomit pomocí " \break "

%vedouci prace
\vedouci{RNDr.}{Jiří Dvořák}{, CSc.}
\citacevedouci{Vedoucí diplomové práce RNDr. Jiří Dvořák, CSc.} % Označení vedoucího práce pro citaci záv. práce. Musí být ukončeno tečkou.

\datumobhajoby{neuvedeno}
\abstrakt{\noindent Tato diplomová práce se zabývá plánováním cesty pro více mobilních robotů. Teoretická část se věnuje popisu navigace robota -- mapování a plánování cesty. Jsou zde popsány vybrané metody umělé inteligence používané při plánování cesty pro více robotů. V praktické části je implementováno simulační prostředí, ve kterém byly vybrané algoritmy srovnány pomocí experimentů.} % Před "\n" vložit další "\n"
\enabstrakt{\noindent This master thesis deals with path planning for multiple mobile robots. The theoretical part describes robot navigation -- mapping and path planning. Selected methods of artificial intelligence used in multi-robot path planning are described. In practical part simulator is implemented, in which selected algorithms were compared using experiments.} % Před "\n" vložit další "\n"
\klicovaslova{\noindent Plánování cesty, mobilní robot, kooperativní plánování} % Před "\n" vložit " \break"
\enklicovaslova{\noindent Path planning, mobile robot, cooperative planning} % Před "\n" vložit " \break"
  
%%
%%   Konec generování údajů
%%


%%
%%   Vlastní vysázení desek umístnit na začátek práce
%%
%\titul% vytiskne titul práce
%\abstrakty% vytiskne stránku s abstrakty
       % Načte data pro vyplnění šablony
%%    \usepackage{fontspec}  % Pro vkládání OTF fontů (vyžaduje titulní list) - nefunguje v pdfLaTeXu
%%
%% Na začátek hlavního souboru (za \begin{document} ) vložte příkazy pro vysázení desek
%%   \titul% vytiskne titul práce
%%   \abstrakty% vytiskne stránku s abstrakty
%%
%% Použité kódování UTF-8
%%
%%% Generování údajů

\fakulta{Fakulta strojního inženýrství}
\enfakulta{Faculty of Mechanical Engineering}
\adresafakulta{Technická 2896/2, 61669 Brno}

\ustav{Ústav automatizace a informatiky}
\enustav{Institute of Automation and Computer Science}

% udaje o autorovi

\autor{Bc.}{Ondřej Sekáč}{}  % Jméno autora, 
    % Tituly vložte samostatně, např. \autor{Ing.}{Petra Smékalová}{}
\autorzkr{Sekáč, O. }
  % bibliografické jméno

\typstudia{N}
  % M, N, B, D
  % M - Magisterské, N - Navazující magisterské, B - Bakalářské, D-Doktorské
  % U typu studia M a N se liší anglický název

\nazev{Plánování cesty pro více robotů} 
  % Ručně můžete dlouhý text zalomit pomocí " \break "
\ennazev{Path planning for multiple robots} 
  % Ručně můžete dlouhý text zalomit pomocí " \break "

%vedouci prace
\vedouci{RNDr.}{Jiří Dvořák}{, CSc.}
\citacevedouci{Vedoucí diplomové práce RNDr. Jiří Dvořák, CSc.} % Označení vedoucího práce pro citaci záv. práce. Musí být ukončeno tečkou.

\datumobhajoby{neuvedeno}
\abstrakt{\noindent Tato diplomová práce se zabývá plánováním cesty pro více mobilních robotů. Teoretická část se věnuje popisu navigace robota -- mapování a plánování cesty. Jsou zde popsány vybrané metody umělé inteligence používané při plánování cesty pro více robotů. V praktické části je implementováno simulační prostředí, ve kterém byly vybrané algoritmy srovnány pomocí experimentů.} % Před "\n" vložit další "\n"
\enabstrakt{\noindent This master thesis deals with path planning for multiple mobile robots. The theoretical part describes robot navigation -- mapping and path planning. Selected methods of artificial intelligence used in multi-robot path planning are described. In practical part simulator is implemented, in which selected algorithms were compared using experiments.} % Před "\n" vložit další "\n"
\klicovaslova{\noindent Plánování cesty, mobilní robot, kooperativní plánování} % Před "\n" vložit " \break"
\enklicovaslova{\noindent Path planning, mobile robot, cooperative planning} % Před "\n" vložit " \break"
  
%%
%%   Konec generování údajů
%%


%%
%%   Vlastní vysázení desek umístnit na začátek práce
%%
%\titul% vytiskne titul práce
%\abstrakty% vytiskne stránku s abstrakty
       % Načte data pro vyplnění šablony
%%    \usepackage{fontspec}  % Pro vkládání OTF fontů (vyžaduje titulní list) - nefunguje v pdfLaTeXu
%%
%% Na začátek hlavního souboru (za \begin{document} ) vložte příkazy pro vysázení desek
%%   \titul% vytiskne titul práce
%%   \abstrakty% vytiskne stránku s abstrakty
%%
%% Použité kódování UTF-8
%%
%%% Generování údajů

\fakulta{Fakulta strojního inženýrství}
\enfakulta{Faculty of Mechanical Engineering}
\adresafakulta{Technická 2896/2, 61669 Brno}

\ustav{Ústav automatizace a informatiky}
\enustav{Institute of Automation and Computer Science}

% udaje o autorovi

\autor{Bc.}{Ondřej Sekáč}{}  % Jméno autora, 
    % Tituly vložte samostatně, např. \autor{Ing.}{Petra Smékalová}{}
\autorzkr{Sekáč, O. }
  % bibliografické jméno

\typstudia{N}
  % M, N, B, D
  % M - Magisterské, N - Navazující magisterské, B - Bakalářské, D-Doktorské
  % U typu studia M a N se liší anglický název

\nazev{Plánování cesty pro více robotů} 
  % Ručně můžete dlouhý text zalomit pomocí " \break "
\ennazev{Path planning for multiple robots} 
  % Ručně můžete dlouhý text zalomit pomocí " \break "

%vedouci prace
\vedouci{RNDr.}{Jiří Dvořák}{, CSc.}
\citacevedouci{Vedoucí diplomové práce RNDr. Jiří Dvořák, CSc.} % Označení vedoucího práce pro citaci záv. práce. Musí být ukončeno tečkou.

\datumobhajoby{neuvedeno}
\abstrakt{\noindent Tato diplomová práce se zabývá plánováním cesty pro více mobilních robotů. Teoretická část se věnuje popisu navigace robota -- mapování a plánování cesty. Jsou zde popsány vybrané metody umělé inteligence používané při plánování cesty pro více robotů. V praktické části je implementováno simulační prostředí, ve kterém byly vybrané algoritmy srovnány pomocí experimentů.} % Před "\n" vložit další "\n"
\enabstrakt{\noindent This master thesis deals with path planning for multiple mobile robots. The theoretical part describes robot navigation -- mapping and path planning. Selected methods of artificial intelligence used in multi-robot path planning are described. In practical part simulator is implemented, in which selected algorithms were compared using experiments.} % Před "\n" vložit další "\n"
\klicovaslova{\noindent Plánování cesty, mobilní robot, kooperativní plánování} % Před "\n" vložit " \break"
\enklicovaslova{\noindent Path planning, mobile robot, cooperative planning} % Před "\n" vložit " \break"
  
%%
%%   Konec generování údajů
%%


%%
%%   Vlastní vysázení desek umístnit na začátek práce
%%
%\titul% vytiskne titul práce
%\abstrakty% vytiskne stránku s abstrakty
       % Načte data pro vyplnění šablony
%%    \usepackage{fontspec}  % Pro vkládání OTF fontů (vyžaduje titulní list) - nefunguje v pdfLaTeXu
%%
%% Na začátek hlavního souboru (za \begin{document} ) vložte příkazy pro vysázení desek
%%   \titul% vytiskne titul práce
%%   \abstrakty% vytiskne stránku s abstrakty
%%
%% Použité kódování UTF-8
%%
%%% Generování údajů

\fakulta{Fakulta strojního inženýrství}
\enfakulta{Faculty of Mechanical Engineering}
\adresafakulta{Technická 2896/2, 61669 Brno}

\ustav{Ústav automatizace a informatiky}
\enustav{Institute of Automation and Computer Science}

% udaje o autorovi

\autor{Bc.}{Ondřej Sekáč}{}  % Jméno autora, 
    % Tituly vložte samostatně, např. \autor{Ing.}{Petra Smékalová}{}
\autorzkr{Sekáč, O.}
  % bibliografické jméno

\typstudia{N}
  % M, N, B, D
  % M - Magisterské, N - Navazující magisterské, B - Bakalářské, D-Doktorské
  % U typu studia M a N se liší anglický název

\nazev{Plánování cesty pro více robotů} 
  % Ručně můžete dlouhý text zalomit pomocí " \break "
\ennazev{Path planning for multiple robots} 
  % Ručně můžete dlouhý text zalomit pomocí " \break "

%vedouci prace
\vedouci{RNDr.}{Jiří Dvořák}{, CSc.}
\citacevedouci{Vedoucí RNDr. Jiří Dvořák, CSc.} % Označení vedoucího práce pro citaci záv. práce. Musí být ukončeno tečkou.

\datumobhajoby{neuvedeno}
\abstrakt{\noindent } % Před "\n" vložit další "\n"
\enabstrakt{\noindent } % Před "\n" vložit další "\n"
\klicovaslova{\noindent } % Před "\n" vložit " \break"
\enklicovaslova{\noindent } % Před "\n" vložit " \break"
  
%%
%%   Konec generování údajů
%%


%%
%%   Vlastní vysázení desek umístnit na začátek práce
%%
%\titul% vytiskne titul práce
%\abstrakty% vytiskne stránku s abstrakty
